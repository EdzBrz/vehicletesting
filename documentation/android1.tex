\subsection{Android}
*namn* is an application for android for which the primary goals were;
%communication goes via a server.
\begin{itemize}
 \item communicate using CoAP over an UDP connection
 \item encode the sent packets with EXI to reduce the size of packets
 \item have an interface capable of controlling multiple devices.
\end{itemize}
% UNDERSÖK OM MER SKA IN HÄR


\subsubsection{How to use}
The application starts off by showing the server menu. One will have to add a server to the list to create a connection with it. For the moment nothing else works.

\subsubsection{How it is supposed to work}

The application start off by showing the server menu. One will have to either add a server or use a saved one. 
Clicking on a server in the list will validate that the server is alive and will perform a CoAP discovery on the server
which will provide a list of devices and services in the device menu. The device menu can then be accessed to use the services available.
The services generelly works as one click on the service item in the list, this will send a message to the server which will respond by contacting the CoAP device. The device will then either update or change something,
it will then send a confirmation with the changes made to server which sends the changes to the application. The application will then use the provided information and update the GUI with the new values. 
The services are generalized to; 
\begin{itemize}
 \item IsOn, provides an item with a checkbox that is checked if a service is on (for example a lamp). Clicking on this should provide a change on the selected device.
 \item GetValue, provides an item which shows the last value provided. Clicking on the item should provide an update.
 \item SendValue, provides an item which have a textbox where one can send a payload to the device. %ska denna även visa vad värdet är för tillfället? tror ej det 
\end{itemize}


