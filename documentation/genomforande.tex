% EXAMPLE: Sections and Paragraphs
% \section		- Level 1
% \subsection		- Level 2
% \subsubsection	- Level 3
% \paragraph		- Level 4
% \subparagraph		- Level 5 
% \section{<NAME_OF_SECTION>}

% EXAMPLE: Include graphics 
% \includegraphics[width=130mm,height=108mm]{intro4.png}

% EXAMPLE: Nested list
%\begin{enumerate}
%\item Nested list
%\begin{enumerate}
%\item
%\item
%\item
%\item
%\item
%\end{enumerate}
%\end{enumerate}

\section{Execution of the project}
\subsection{Scrum and how it has been used}
It was decided back in november that the entire project would be divded into three sprints.
The exact dates were to be decided in the beginning of each sprint.
In cooperation with the client the scope of the project and the scope of the first sprint was decided upon in november.
During the first projectmeeting the first sprint goal was divided into eight sprint stories.
It soon became clear that those eight stories were way to big, at the end of the sprint none of the stories had been finished.

Lesson learnt, the second sprint was divided into smaller stories which gave immidiate result when the first 69 sprint story points finished during the second sprint.

To decide upon size for each sprint story, for the second and third sprint, "planning poker" \cite[p.~42]{kniberg07} was used.
For every sprint story each project member everyone wrote down an estimate on the scope for each story.
With planning poker it became clear that each project member had a different vision for each story.
A short discussion after each estimate made it more clear on how big the scope was, an agreement was usually made within a few minutes.

%
% TODO: Skriv hur vi har använt Scrum och relatera till våra referenser \cite{kniberg07}
%
\subsection{One project, three sprint goals}
TODO: Vi delade in oss i tre olika grupper, mulle, server och android under ett skype-möte.
TODO: Efter sprint planning första två gångerna var det upp till varje "grupp" att fördela uppgifter.
TODO: Under andra sprinten flyttade vi resurser från server-delen då den var långt före de andra delarna.
TODO: Flyttade tillbaka en resurs i början av tredje sprinten eftersom det fortfarande var svåra flaskhalsar vi satt fast med på Mullen...onödigt att tre sitter fast på samma ställe.
TODO: Sista sprinten tog var och en direkt en uppgift från sprint backlog och satte den mer tydlig som "sin". Detta ledde till viss förbättring men eftersom vi inte har dagliga scrum möten så ger det inte en daglig uppdatering/reflektion över hur det faktiskt går...sprint backloggen blir inte en del av vardagen.
\subsection{Individual time monitoring and speed}
\subsubsection{Sophia Bergendahl}
\subsubsection{Edvinn Bruun}
\subsubsection{William Gustafsson}
\subsubsection{Christoffer Holmstedt}
What I've learnt during this project is that there is no way to estimate a reasonable time without knowing something about the topic of concern beforehand.
I've had several stories assigned to me in the topic of linux installation and configuration e.g. installing a webserver and a mysql database.
Both the webserver and database I've installed several times before and knew exactly which steps I was supposed to do to get it up and running as soon as possible.
Of course this made it really easy to estimate an expected time for these stories, or at least most of them.
One story was about installing Ubuntu.
Even though I have installed Ubuntu several times before I didn't anticipate that an entire installation could be so slow, I had simply missed to take into account that we were running our server on a very old machine.
The time estimate for this story was of by 100\%.

The other stories that didn't go so well was mainly about python programming and working with the coapy server implementation. 
It took generally more time than I expected to get started.
In the future I will increase the expected time before I get going e.g. the time to read about a new software I will work with and the time to try out the current functionality.
Though I will keep my estimate for the actual coding parts.

In the future when it's up to me to do another estimate concerning a topic I've never worked with I hope I have a few colleagues with knowledge in that area to help me out with my first estimate.
As long as I guess an estimate instead of chosing "I don't know" I will improve and eventually I will now my "speed".

\subsubsection{Marcus Rådman}
\subsubsection{Kristoffer Svensson}
\paragraph{Assignments}
\begin{enumerate}
\item{"Restructure the loading of new services for the python server"}

I had some previous experience with python on top of already been dealing with the actual server implemention we decided to use prior to dealing with this issue so I was pretty certain I would be able to complete it in about 3 hours.
Without any major issues I managed to complete it in roughly 3 hours and 30 minutes. The question at that point was rather if the solution was adequate, which after some consultation it was deemed to be.

\item {"Figure out how service discovery works in CoAP"}

The estimated time for this was set between 8 and 10 hours. Reason being that it was new ground for me but still didn't feel like it was that big of
an issue. It turned out to be a relatively small fix after 6 hours of work. The issue of whether the solution was good enough or not for the end-purposes was brought up again but it was decided that it was good enough for the time being.
With that said, I doubt that the estimated 8-10 hours would have been sufficient if a end-purpose qualifying solution was to be made. 

\item {"Implement EXIficient"}

Originally the time for this assignment was set to 20 hours, when the assignment was still "Implement EXI parser". In that situation we assumed that there was more work to be done making the parser from the ground up.

When EXIficient was discovered we found out that it had a ready-made demo that almost suited our needs. It turned out that the actual time needed to get something that would do the job for us would be around 10 hours modifying that demo, which is exactly what ended up happening.

\item {"Individual documention"}

I assumed about 2-4 hours for the actual writing with an added hour or two for the research needed to know what was to be included in the report. In the end, this all came down to about 4-5 hours which is well within the expected range.

\end{enumerate}

\paragraph{General thoughts concerning time estimation}

I feel as if my personal time estimations, contrary to what I expected, have been quite accurate. I reckon the reason to this being that I had the fortune of having some kind of knowledge concerning the assignments prior to doing them. Without any idea on the scope or size of the assignments I think that the estimations wouldn't have been this accurate.

The estimations themselves have been useful tools for knowing when to ask for help or guidance. When you're approaching a time limit and you know that you're kind of stuck that's a really good indicator that help is needed. I think it's been quite a useful tool, at least for me personally.

\subsubsection{Ludwig Thurfjell}
\subsection{Reflection about Scrum usage during this project}
