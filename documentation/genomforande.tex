% EXAMPLE: Sections and Paragraphs
% \section		- Level 1
% \subsection		- Level 2
% \subsubsection	- Level 3
% \paragraph		- Level 4
% \subparagraph		- Level 5 
% \section{<NAME_OF_SECTION>}

% EXAMPLE: Include graphics 
% \includegraphics[width=130mm,height=108mm]{intro4.png}

% EXAMPLE: Nested list
%\begin{enumerate}
%\item Nested list
%\begin{enumerate}
%\item
%\item
%\item
%\item
%\item
%\end{enumerate}
%\end{enumerate}

\section{Execution of the project}
\subsection{Scrum and how it has been used}
It was decided back in november that the entire project would be divded into three sprints.
The exact dates were to be decided in the beginning of each sprint.
In cooperation with the client the scope of the project and the scope of the first sprint was decided upon in november.
During the first projectmeeting the first sprint goal was divided into eight sprint stories.
It soon became clear that those eight stories were way to big, at the end of the sprint none of the stories had been finished.

Lesson learnt, the second sprint was divided into smaller stories which gave immidiate result when the first 69 sprint story points finished during the second sprint.

To decide upon size for each sprint story, for the second and third sprint, "planning poker" \cite[p.~42]{kniberg07} was used.
For every sprint story each project member everyone wrote down an estimate on the scope for each story.
With planning poker it became clear that each project member had a different vision for each story.
A short discussion after each estimate made it more clear on how big the scope was, an agreement was usually made within a few minutes.

%
% TODO: Skriv hur vi har använt Scrum och relatera till våra referenser \cite{kniberg07}
%
\subsection{One project, three sprint goals}
TODO: Vi delade in oss i tre olika grupper, mulle, server och android under ett skype-möte.
TODO: Efter sprint planning första två gångerna var det upp till varje "grupp" att fördela uppgifter.
TODO: Under andra sprinten flyttade vi resurser från server-delen då den var långt före de andra delarna.
TODO: Flyttade tillbaka en resurs i början av tredje sprinten eftersom det fortfarande var svåra flaskhalsar vi satt fast med på Mullen...onödigt att tre sitter fast på samma ställe.
TODO: Sista sprinten tog var och en direkt en uppgift från sprint backlog och satte den mer tydlig som "sin". Detta ledde till viss förbättring men eftersom vi inte har dagliga scrum möten så ger det inte en daglig uppdatering/reflektion över hur det faktiskt går...sprint backloggen blir inte en del av vardagen.
\subsection{Individual time monitoring and speed}
\subsubsection{Sophia Bergendahl}
\subsubsection{Edvinn Bruun}
\subsubsection{William Gustafsson}
\subsubsection{Christoffer Holmstedt}
Exempelvis kan lista användas om man vill. Jag, christoffer, kommer inte använda det tror jag.
\begin{enumerate}
\item TODO: Hur man citerar till specifik sida i kurslitteraturen. \cite[p.~42]{kniberg07}
\item TODO: Hur man citerar utan sidhänvisning \cite{kniberg07}
\item Third sprint story
\end{enumerate}
\subsubsection{Marcus Rådman}
\subsubsection{Kristoffer Svensson}
\paragraph{Punkter}
\begin{enumerate}
\item{"Strukturera om inladdning av nya services i python servern"}

Jag hade lite förkunskap av python, samt att jag även hade suttit en del med server implementation i fråga, så jag trodde inte att det skulle ta mer än 3 timmar. 
Relativt nära till det så avslutade jag uppgiften på 3 timmar och 30 minuter utan att ha haft några större problem. Frågan var snarare om lösningen var tillräcklig, vilket
den efter efterfrågan dömdes vara.

\item {"Ta reda på hur discovery av tjänster fungerar i coap"}

Det estimerade tiden sattes till 8-10 timmar. Detta med anledning av att det vara ganska nytt område för mig men samtidigt inte kändes som att det skulle vara någon
större del. Det visade sig vara en mindre fix efter 6 timmars arbete, men även här så är det en frågan om hur man vill att det ska fungera i en slutimplementation. 
Funktionaliteten som uppnåddes dömdes dock vara tillräcklig vid detta tillfälle. Med det sagt så tror jag inte att de 8-10 timmarna hade räckt till om man skulle få
det att fungera perfekt med både Copper och framtida mulle noders krav.

\item {"Implement EXIficient"}

Tiden hade ursprungligen estimerats till 20 timmar, då punkten fortfarande var "Implement EXI parser". I det läget så antog man alltså att man skulle behöva göra något lite
mer från grunden. 

Då EXIficient upptäcktes och man även nöjde sig med funktionaliteten som kom med i deras demo implementation så visade det sig att tiden som faktiskt behövdes för
att få någonting användbart snarare var runt 10 timmar, vilket även var vad det tog i slutändan.

\item {"Individual documention"}

Utgick med 2-4 timmar för skrivandet samt någon extra timme för att få koll på vad som faktiskt skall ingå i rapporten. Slutade runt 4-5 timmar totalt vilket var inom förväntningarna.

\end{enumerate}

\paragraph{Generell tidsestimering}

Jag känner att min personliga tidsestimering, mot mina förväntningar, har varit ganska träffsäkra. Det känns även som att det till stor del grundar sig i hur mycket du vet om uppgiften.
Utan någon som helst förkunskap så blir det otroligt svårt att kunna estimera någonting, men jag har haft turen att ha lite insikt på någon aspekt av varje punkt jag arbetat med.

Tidsestimeringen i sig har varit användbar som ett verktyg för att veta när man behöver ta hjälp utifrån och har, enligt mig, varit väldigt nyttig.

\end{document}
\subsubsection{Ludwig Thurfjell}
\subsection{Reflection about Scrum usage during this project}
