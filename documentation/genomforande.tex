% EXAMPLE: Sections and Paragraphs
% \section		- Level 1
% \subsection		- Level 2
% \subsubsection	- Level 3
% \paragraph		- Level 4
% \subparagraph		- Level 5 
% \section{<NAME_OF_SECTION>}

% EXAMPLE: Include graphics 
% \includegraphics[width=130mm,height=108mm]{intro4.png}

% EXAMPLE: Nested list
%\begin{enumerate}
%\item Nested list
%\begin{enumerate}
%\item
%\item
%\item
%\item
%\item
%\end{enumerate}
%\end{enumerate}

\section{Execution of the project}
\subsection{SCRUM and how we have used it}
We started of in november with deciding how many sprints we will have and between which dates.
The entire project of 17 weeks half time, was divided into three sprints.
Before our first sprint planning meeting we sat down with our client to get a feeling on what we were supposed to do.
During our first sprint planning meeting we divided the sprint goal into smaller sprint stories, due to all of us being new to Scrum and not having a clear picture on how much coding that was needed we ended up with 8 sprint stories.
Due to the small number of sprint stories in relation to the big sprint goal we ended up not finishing any sprint story.
Lesson learnt before the second sprint were that we tried to divide the sprint goal into even smaller stories.
During the second sprint we finished our first 69 story points. 

To decide upon size for each sprint story(scope) for the second and third sprint we used "planning poker" \cite[p.~42]{kniberg07}.
With pen and paper we wrote down our estimate on how "big" we expected each story to be.
Some stories we had completely different picture of how big its scope was which led to a short discussion about it.
In the end we had the same mental picture of each sprint story, a lesson learnt is that the sprint stories still was to big.
%
% TODO: Skriv hur vi har använt Scrum och relatera till våra referenser \cite{kniberg07}
%
\subsection{One project, three sprint goals}
TODO: Vi delade in oss i tre olika grupper, mulle, server och android under ett skype-möte.
TODO: Efter sprint planning första två gångerna var det upp till varje "grupp" att fördela uppgifter.
TODO: Under andra sprinten flyttade vi resurser från server-delen då den var långt före de andra delarna.
TODO: Flyttade tillbaka en resurs i början av tredje sprinten eftersom det fortfarande var svåra flaskhalsar vi satt fast med på Mullen...onödigt att tre sitter fast på samma ställe.
TODO: Sista sprinten tog var och en direkt en uppgift från sprint backlog och satte den mer tydlig som "sin". Detta ledde till viss förbättring men eftersom vi inte har dagliga scrum möten så ger det inte en daglig uppdatering/reflektion över hur det faktiskt går...sprint backloggen blir inte en del av vardagen.
\subsection{Individual time monitoring and our "speed"}
\subsubsection{Sophia Bergendahl}
\subsubsection{Edvinn Bruun}
\subsubsection{William Gustafsson}
\subsubsection{Christoffer Holmstedt}
\begin{enumerate}
\item First sprint story
\item Second sprint story
\item Third sprint story
\item Fourth sprint story
\item Fifth sprint story
\end{enumerate}
\subsubsection{Marcus Rådman}
\subsubsection{Kristoffer Svensson}
\subsubsection{Ludwig Thurfjell}
\subsection{Reflection and discussion about SCRUM for our project}
