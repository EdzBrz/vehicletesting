% EXAMPLE: Sections and Paragraphs
% \section		- Level 1
% \subsection		- Level 2
% \subsubsection	- Level 3
% \paragraph		- Level 4
% \subparagraph		- Level 5 
% \section{<NAME_OF_SECTION>}

% EXAMPLE: Include graphics 
% \includegraphics[width=130mm,height=108mm]{intro4.png}

% EXAMPLE: Nested list
%\begin{enumerate}
%\item Nested list
%\begin{enumerate}
%\item
%\item
%\item
%\item
%\item
%\end{enumerate}
%\end{enumerate}

\section{Conclusions}
This project started out to be of reasonable size for our project group but ended up way to big mainly due to some persistent bottlenecks in some components.
This lead to no component (Mulle, CoAP server and Android application) being fully functional in the end.
This is exactly what Scrum tries to prevent, as discussed earlier in this report a better solution to improve quality would have been to do a few more, but smaller sprints.

Concerning usability and future improvements there is alot more work that needs to be done.
The current state of the the software is that all components are lacking small parts in different areas, this makes it unusable in it's current state.
Future improvements should start with taking a step back and really rethink what the goal is and start working with one component at a time.
Highest priority should be to get the Mulle operational and able to communicate with it's surroundings.
Without sensors there is no data to analyze or to reconfigure "on the fly".
