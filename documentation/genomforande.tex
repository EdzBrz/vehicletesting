% EXAMPLE: Sections and Paragraphs
% \section		- Level 1
% \subsection		- Level 2
% \subsubsection	- Level 3
% \paragraph		- Level 4
% \subparagraph		- Level 5 
% \section{<NAME_OF_SECTION>}

% EXAMPLE: Include graphics 
% \includegraphics[width=130mm,height=108mm]{intro4.png}

% EXAMPLE: Nested list
%\begin{enumerate}
%\item Nested list
%\begin{enumerate}
%\item
%\item
%\item
%\item
%\item
%\end{enumerate}
%\end{enumerate}

\section{Execution of the project}
\subsection{Scrum and how it has been used}
It was decided back in november that the entire project would be divded into three sprints.
The exact dates were to be decided in the beginning of each sprint.
In cooperation with the client the scope of the project and the scope of the first sprint was decided upon in november.
During the first projectmeeting the first sprint goal was divided into eight sprint stories.
It soon became clear that those eight stories were way to big, at the end of the sprint none of the stories had been finished.

Lesson learnt, the second sprint was divided into smaller stories which gave immidiate result when the first 69 sprint story points finished during the second sprint.

To decide upon size for each sprint story, for the second and third sprint, "planning poker" \cite[p.~42]{kniberg07} was used.
For every sprint story each project member wrote down an estimate on the scope for each story.
With planning poker it became clear that each project member had a different vision for each story.
A short discussion after each estimate made it more clear on how big the scope was, an agreement was usually made within a few minutes.

\subsection{One project, three sprint goals}
As mentioned earlier the project was divided into three sprints.
This meant that three different sprint goals had to be divided into smaller sprint stories which in turn had to be assigned to a project member.
Due to all project members being new to most of the tasks at hand the first team division was made with focus on components \cite[p.~106]{kniberg07}.
The goal with this was that each smaller team within the project could sit together and dive deep into their specific part such as the Mulle, the server parts or the android code.
Later on a split into cross-component teams \cite[p.~107]{kniberg07} was aimed for but due to some persistent bottlenecks in some components this was never done.

For all sprints the sprint planning meeting were used to categorize each sprint story into the different components (Mulle, Server, Android).
It was then up to each component based team to split their stories between themselves.
This ended being a very flexible solution, in some cases, to flexible when a team didn't use the Scrumboard online at Scrumdo.com the team could wander off from the sprint story they were supposed to work on.
For the last sprint everyone got at least one sprint story assigned to themselves directly during the sprint planning meeting.
This was made to put more focus on using the Scrumboard at Scrumdo.com.

A move to cross-component teams was never made instead an attempt to increase speed for the Mulle and the Android component was made by moving one team member from the server team to the Mulle team and another to the Android team.
The server component at this time was way ahead of the other components.
Another week later it became clear that the additional team member for Mulle team was not needed, cause the problem was still a bottleneck that only one or two team members could work on at a single point in time, a move back to the server team was made.

\subsection{Individual time monitoring and speed}
\subsubsection{Sophia Bergendahl}
\subsubsection{Edvin Bruun}
\subsubsection{William Gustafsson}
\subsubsection{Christoffer Holmstedt}
Exempelvis kan lista användas om man vill. Jag, christoffer, kommer inte använda det tror jag.
\begin{enumerate}
\item TODO: Hur man citerar till specifik sida i kurslitteraturen. \cite[p.~42]{kniberg07}
\item TODO: Hur man citerar utan sidhänvisning \cite{kniberg07}
\item Third sprint story
\end{enumerate}
\subsubsection{Marcus Rådman}
\subsubsection{Kristoffer Svensson}
\subsubsection{Ludwig Thurfjell}
\subsection{Reflection about Scrum usage during this project}
Alot has been learnt during this project and to keep it short the following lessons learnt and improvements that can be made are a chosen few.

At the end of all projects when the deadline is closing in the pace of the development often increases.
A downfall of this is that when the speed increases the quality of the code often decreases.
Scrum is a solution to this problem with the main goal of keeping a steady development pace and always keeping the quality of the code as good as possible above some minimum criteria.
With to long sprints, projects will still end up with a big deadline, this is what happened in this project.
In total the project lasted about 17 weeks including the winter holidays.
Instead of three sprints where each lasted about five weeks a project split into smaller sprint would have been better.
If time travel was possible this project would have had two smaller sprints in the beginning each would have lasted for two weeks.
The first sprint with goal of configuring all required software for everyone such as setting up git, the different IDEs, gcc and other required software/tools.
The second sprint with the goal of understanding what was available at the time and get all basic functionality working.
If the project had found any big bottlenecks at the end of the second sprint that would be the point in time to halt the project and really rethink what to focus on.
The remaining weeks should have been divided into three evenly sized sprints.

Another big improvement could have been made in the relation between Scrum team and product owner.
From the beginning it was unclear who was the product owner out of three supervisors/clients, this should have been defined to one single person as early as possible before continuing with any other work.
The absence of the product owner before and during our sprint planning meetings resulted in alot of confusion when it was up to the team to decide an estimate for each sprint story.
The lesson learnt from this is that without a product owner the scrum team will fumble in blindness forever or as Kniberg writes \cite[p.~25]{kniberg07} {\em "...each story contains three variables that are highly dependent on each other"}.
There is no way the team can choose a good estimate when there is no product owner that has already chosen an importance and scope for each sprint story to start with and is available to change that during the course of a sprint planning meeting.
This also lead to the project growing in scope without any clear bounds.

The last and perhaps the most important improvement that will be mentioned is the importance of having a team member taking the role as Scrum master.
Without the Scrum master during the first sprint everyone was on their own.
For the second and third sprint the Scrum master role was appointed to one team member.
This improved the communication within the Scrum team alot and also made it possible for individual team members to have someone to go to in case of general questions and/or other problems.

%
% Andra kommentarer som kan vara värda att ta med om Scrum och hur det 
% har använts i detta projekt om det får plats. Texterna är redan lite för långa.
%
%TODO: Tydligare sprint mål för varje sprint från kravställare. En mening som beskriver syftet med respektive sprint.
%TODO: Tydligare demo av varje punkt
%TODO: Vi skulle ha satt ner foten på hur många delar vi skulle jobba med samtidigt. Startat med Mulle och Server enbart...sedan android i mån av tid för att få mer fokus i projektet, vi var trots allt bara sju personer.
%TODO: Ingen "experthjälp" med android tillgänglig. I detta läge gör det stor skillnad om man är två, tre eller fyra. Varje programmerare ökar hastigheten med mer än bara sin tid hen kan lägga till eftersom felsökning oftast går betydligt snabbare för fler personer än enbart två. Många fel beror också på utvecklingsmiljön, om man är fler med samma miljlö så är det större chans att någon har stött på problemet tidigare och har en lösning.
%TODO: Reflektion över flytten av individuella team members server till mulle och android. Väldigt "dyrt" i uppstartskostnad.
