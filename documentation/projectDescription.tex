% EXAMPLE: Sections and Paragraphs
% \section		- Level 1
% \subsection		- Level 2
% \subsubsection	- Level 3
% \paragraph		- Level 4
% \subparagraph		- Level 5 
% \section{<NAME_OF_SECTION>}

% EXAMPLE: Include graphics 
% \includegraphics[width=130mm,height=108mm]{intro4.png}

% EXAMPLE: Nested list
%\begin{enumerate}
%\item Nested list
%\begin{enumerate}
%\item
%\item
%\item
%\item
%\item
%\end{enumerate}
%\end{enumerate}

\section{Project Description}
\subsection{Background}
Luleå University of Technology conducts research on lowpower wireless microprocessors called "Mulle". 
These microprocessors can be used for various things depending on which type of sensors you connect to it, everything from measuring temperature or vibrations in a car to analyzing the quality of the road that you drive on.

Every year northern parts of Sweden are used for testing cars during winter conditions.
To test a car you first decide what you want to test, then you test with local sensors logging within the car.
When enough data is collected you return back home.
At the testing facility the data is now available for analysis.
Depending on the results from the previous runs you might want to test some parts in more detail so you re-configure all sensors and go out for another test run.

This process is time consuming when you need to return to testing facility to be able to analyze and re-configure all sensors.
In todays society most computers are connected to internet and/or other private networks, most of these computers have the ability to be remotely configured and maintained.
The goal with this project is to be able to analyze data from sensors in realtime and re-configure them on the fly while testing is in progress.
\subsection{Project Targets}
\begin{enumerate}
\item Be able to send live sensor data from multiple "Mulles" to an online logging server/service.
\item Be able to read sensor data on the web with both a PC (web browser) and through an Android mobile device.
\item Be able to re-configure the sensors through a web interface and through an Android mobile device.
\end{enumerate}


%
%
% Kom-ihåg kommentarer nedanför detta. Ta bort när de inte behövs längre.
%
%
\pagebreak
\begin{enumerate}
\item Bakgrund och övergripande beskrivning av projektuppgift
\begin{enumerate}
\item Övergripande överenskommelse med kravställare, används också i Resultatrapport
\end{enumerate} 
\item Beskrivning av uppgifter, prioritet, estimerad tid (verklig tid), förändringar
\begin{enumerate}
\item Product backlogg 
\item Sprint backlogg - ska diskuteras men själva backloggen ska inte med.
\end{enumerate} 
\end{enumerate}
