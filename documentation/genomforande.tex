% EXAMPLE: Sections and Paragraphs
% \section		- Level 1
% \subsection		- Level 2
% \subsubsection	- Level 3
% \paragraph		- Level 4
% \subparagraph		- Level 5 
% \section{<NAME_OF_SECTION>}

% EXAMPLE: Include graphics 
% \includegraphics[width=130mm,height=108mm]{intro4.png}

% EXAMPLE: Nested list
%\begin{enumerate}
%\item Nested list
%\begin{enumerate}
%\item
%\item
%\item
%\item
%\item
%\end{enumerate}
%\end{enumerate}

\section{Execution of the project}
\subsection{Scrum and how it has been used}
It was decided back in november that the entire project would be divded into three sprints.
The exact dates were to be decided in the beginning of each sprint.
In cooperation with the client the scope of the project and the scope of the first sprint was decided upon in november.
During the first projectmeeting the first sprint goal was divided into eight sprint stories.
It soon became clear that those eight stories were way to big, at the end of the sprint none of the stories had been finished.

Lesson learnt, the second sprint was divided into smaller stories which gave immidiate result when the first 69 sprint story points finished during the second sprint.

To decide upon size for each sprint story, for the second and third sprint, "planning poker" \cite[p.~42]{kniberg07} was used.
For every sprint story each project member everyone wrote down an estimate on the scope for each story.
With planning poker it became clear that each project member had a different vision for each story.
A short discussion after each estimate made it more clear on how big the scope was, an agreement was usually made within a few minutes.

%
% TODO: Skriv hur vi har använt Scrum och relatera till våra referenser \cite{kniberg07}
%
\subsection{One project, three sprint goals}
TODO: Vi delade in oss i tre olika grupper, mulle, server och android under ett skype-möte.
TODO: Efter sprint planning första två gångerna var det upp till varje "grupp" att fördela uppgifter.
TODO: Under andra sprinten flyttade vi resurser från server-delen då den var långt före de andra delarna.
TODO: Flyttade tillbaka en resurs i början av tredje sprinten eftersom det fortfarande var svåra flaskhalsar vi satt fast med på Mullen...onödigt att tre sitter fast på samma ställe.
TODO: Sista sprinten tog var och en direkt en uppgift från sprint backlog och satte den mer tydlig som "sin". Detta ledde till viss förbättring men eftersom vi inte har dagliga scrum möten så ger det inte en daglig uppdatering/reflektion över hur det faktiskt går...sprint backloggen blir inte en del av vardagen.
\subsection{Individual time monitoring and speed}
\subsubsection{Sophia Bergendahl}
\newpage
\subsubsection{Edvin Bruun}
Throughout the project I, as well as my fellow group members, have been gradually learning to work with the Scrum project-model.
This work-model includes a tactic for distributing work called ''stories''. These stories are given an time estimate and a actual time when they are done, 
and following this I will explain three stories that I've encountered in this project. 
For convenience I've chosen one story from each iteration to roughly show how the work-model was more and more used. As a quick reminder, my stories have revolved around getting the Mulle-communication to work.
\\\\
To start things off, I chose the first story that I had, which was titled ''CoAP communication over bluetooth from Mulle to server''.
This story was estimated to take the entire first sprint which was roughly eight weeks long. In this story there was a pretty hefty start-up time included, 
as most of us were clueless as to how much time we would spend on learning the new technologies in each of our assignment-fields(Mulle, Server, Android).
When the eight weeks were up, the progress on the story was horrible to say the least. During our first sprint we struggled with issues that were very much 
out of the scope of what we should have been doing. The reason for this was mainly because we needed to get these things working before we could start progressing on the actual story.
However when the time was up we agreed that we'd have to take a different route to achieve the communication so the actual time for this story was pretty accurate.
\\\\
Moving on to the next story, this story took place in the second iteration and revolved around sending UDP packets from the Mulle to the server.
The point of this story was to check if we could achieve the simplest of communication(with our new approach) and then build from there. This story was scheduled to run over 
a week  which also was our estimate. The actual time for this story was however a few days over the estimate. The reason for this was that at this point we were working on multiple stories as
well as some issues with our testing methods. However looking back on the things that caused the delay I don't see how we could have avoided them.
\\\\
The third story to be explained in this documentation is a story that was very small and which took place in the third iteration.
The task in the story was to get the Mulle to run a certain function every time it received a UDP package. This was estimated at roughly ten hours, 
were as the actual time it took was closer to five. The reason for the shorter time than anticipated was that I had gained knowledge how to do this 
indirectly through another story which I'd worked on earlier.
\subsubsection{William Gustafsson}
\subsubsection{Christoffer Holmstedt}
Exempelvis kan lista användas om man vill. Jag, christoffer, kommer inte använda det tror jag.
\begin{enumerate}
\item TODO: Hur man citerar till specifik sida i kurslitteraturen. \cite[p.~42]{kniberg07}
\item TODO: Hur man citerar utan sidhänvisning \cite{kniberg07}
\item Third sprint story
\end{enumerate}
\subsubsection{Marcus Rådman}
\subsubsection{Kristoffer Svensson}
\subsubsection{Ludwig Thurfjell}
\subsection{Reflection about Scrum usage during this project}
