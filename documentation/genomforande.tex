% EXAMPLE: Sections and Paragraphs
% \section		- Level 1
% \subsection		- Level 2
% \subsubsection	- Level 3
% \paragraph		- Level 4
% \subparagraph		- Level 5 
% \section{<NAME_OF_SECTION>}

% EXAMPLE: Include graphics 
% \includegraphics[width=130mm,height=108mm]{intro4.png}

% EXAMPLE: Nested list
%\begin{enumerate}
%\item Nested list
%\begin{enumerate}
%\item
%\item
%\item
%\item
%\item
%\end{enumerate}
%\end{enumerate}

\section{Execution of the project}
\subsection{Scrum and how it has been used}
It was decided back in November that the entire project would be divded into three sprints.
The exact dates were to be decided in the beginning of each sprint.
In cooperation with the client the scope of the project and the scope of the first sprint was decided upon in November.
During the first project meeting the first sprint goal was divided into eight sprint stories.
It soon became clear that those eight stories were way to big, at the end of the sprint none of the stories had been finished.

Lesson learnt, the second sprint was divided into smaller stories which gave immidiate result when the first 69 sprint story points finished during the second sprint.

To decide upon scope for each sprint story, for the second and third sprint, "planning poker" \cite[p.~42]{kniberg07} was used.
For every sprint story each project member wrote down an estimate on the scope for each story.
With planning poker it became clear that each project member had a different vision for each story.
A short discussion after each estimate made it more clear on how big the scope was, an agreement was usually made within a few minutes.

\subsection{One project, three sprint goals}
As mentioned earlier the project was divided into three sprints.
This meant that three different sprint goals had to be divided into smaller sprint stories which in turn had to be assigned to a project member.
Due to all project members being new to most of the tasks at hand the first team division was made with focus on components \cite[p.~106]{kniberg07}.
The goal with this was that each smaller team within the project could work together and dive deep into their specific part such as the Mulle, the server parts or the android code.
Later on, a split into cross-component teams \cite[p.~107]{kniberg07} was aimed for but due to some persistent bottlenecks in some components this was never done.

For all sprints the sprint planning meeting was used to categorize each sprint story into the different components (Mulle, Server, Android).
It was then up to each component based team to split their stories between themselves.
This ended being a very flexible solution and in some cases, too flexible. 
This lead to teams not using the Scrumboard online at Scrumdo.com and wander off from the sprint story they were supposed to work on.
For the last sprint everyone got at least one sprint story assigned to themselves directly during the sprint planning meeting.
This was made to put more focus on using the Scrumboard at Scrumdo.com.

A move to cross-component teams was never made, instead an attempt to increase development speed for the Mulle and the Android components were made by moving one team member from the server team to the Mulle team and another to the Android team.
The server component at this time was ahead of the other components.
A week later it became clear that the additional team member for the Mulle team was not needed, because the problem was still a bottleneck that only one or two team members could work on at a single point in time, a move back to the server team was made.

\subsection{Individual time monitoring and speed}
\subsubsection{Sophia Bergendahl}

\paragraph{Generally about time estimation}
Time estimation is something that I have found difficult to do, and it is something that shows in the three stories below. In the stories I have tried to find reasons to why the time 
estimation was inaccurate. Generally I believe the estimation was inaccurate because of inexperience, when having done something once it is easier to estimate time, but if it is the first time it 
is more difficult. If I had known more about the stories before doing them and if I in the past had reflected more about how fast or slow I work, I believe my time estimation would have been 
better.


\paragraph{Three stories}
\begin{enumerate}
\item{"Create a PHP-web page that presents information stored in the database."}

I had no experiences in creating a web page so I estimated that it would take me about eight hours, first to learn the basics about html, php and css, and then to implement it. It took me
about 12 hours, I believe it took longer because of the problems I had with retrieving data from the database and that I decided to split the php file containing the finished web page
into three parts header, main and footer to make it easier to add to the page. 

\item {"Add a button to the web page that can call Mulle-services on the server"}

I had some experience with php so I estimated that it would take me about four hours, which was about what it took for me to create a php form, understand how it works and connect it to a
python script that can call Mulle-services. I believe that since I had worked with php before it made it easier to estimate time. Furthermore if the  button I made had not been turn on/off
lamp, whose services is not implemented on the server, it would have taken me longer to finish because the python script I created need information about the service to call to work
fully. 

\item {"Create a database that keeps track of connected/active devices."}

I estimated that it would take me about eight hours, since I had no experience with databases, more than retrieving information from one. It took me about six hours, the reason to why it was
created quicker than expected was because I had few problems and I did not have the possibility to test it with a Mulle due to its communication problems.
 
\end{enumerate}	
\subsubsection{Edvin Bruun}
Throughout the project I, as well as my fellow group members, have been gradually learning to work with the Scrum project-model.
This work-model includes a tactic for distributing work called ''stories''. These stories are given a time estimate and an actual time when they are done, 
and following this I will explain three stories that I've encountered in this project. 
For convenience I've chosen one story from each iteration to roughly show how the work-model was more and more used. As a quick reminder, my stories have revolved around getting the Mulle-communication to work.

To start things off, I chose the first story that I had, which was titled ''CoAP communication over bluetooth from Mulle to server''.
This story was estimated to take the entire first sprint which was roughly eight weeks long. In this story there was a pretty hefty start-up time included, 
as most of us were clueless as to how much time we would spend on learning the new technologies in each of our assignment-fields(Mulle, Server, Android).
When the eight weeks were up, the progress on the story was horrible to say the least. During our first sprint we struggled with issues that were very much 
out of the scope of what we should have been doing. The reason for this was mainly because we needed to get these things working before we could start progressing on the actual story.
However when the time was up we agreed that we'd have to take a different route to achieve the communication so the actual time for this story was pretty accurate.

Moving on to the next story, this story took place in the second iteration and revolved around sending UDP packets from the Mulle to the server.
The point of this story was to check if we could achieve the simplest of communication(with our new approach) and then build from there. This story was scheduled to run over 
a week  which also was our estimate. The actual time for this story was however a few days over the estimate. The reason for this was that at this point we were working on multiple stories as
well as some issues with our testing methods. Looking back on the things that caused the delay I don't see how we could have avoided them.

The third story to be explained in this documentation is a story that was very small and which took place in the third iteration.
The task in the story was to get the Mulle to run a certain function every time it received a UDP package. This was estimated at roughly ten hours, 
were as the actual time it took was closer to five. The reason for the shorter time than anticipated was that I had gained knowledge how to do this 
indirectly through another story which I'd worked on earlier.

\subsubsection{William Gustafsson}
One of the main things about android that I've learnt is that it takes time, there's alot of documention that needs to be read and most of the guides already assumes that you know everything about android, except the part that you're looking up.

\noindent In the first iteration I was working with setting up everything needed to program Android. This means getting eclipse and it's plugins, android adb and all the other software working. 
Then I threw myself in and started working with the Bluetooth part of the application, my thoughts on this was that it wouldn't be so hard to get working. 
Reality then showed me otherwise and it both took alot more time than anticipated and didn't turn out as I had wished,
this mostly since I realised Android wasn't capable of doing the features wanted without changing things in the android operating system. This howeber gave me a grip of how I was supposed to
do when programming android. So the second iteration I used some time doing the basic android stuff like the manifest and getting the xml and logic of the application to work together. 
This would have taken alot more time without the help of Karl Öhman, fourth year student at Luleå University of Technology.             
\smallskip

\noindent Another story of the second iteration that actually was finished within the expected time was to design how the GUI would look, I did a couple of designs and asked around for which seemed like the best option, and
the result of that choice has been alot of the work done since. So since the basic Android things were working and the GUI and logic of the lists in 
the application were pretty close tied in Android programming I took the responsibility for those stories. They've been pretty time consuming since there has been alot of going back and forth in them since the 
basic components weren't capable of producing the result sought.

And now lastly, about my experience with this. Firstly I thought scrum was not really necessary and only a waste of time. But using it for a while I realised that knowing your speed and being able to break down problems in the way scrum was meant to be used, are really useful tools that can be used to speed up the project alot. About the android, well some of the things I've learnt is that before you can start coding the project there are alot of small things that must be done, of which most are kind of hard to find unless you know what you're searching for.

\subsubsection{Christoffer Holmstedt}
What I've learnt during this project is that there is no way to estimate a reasonable time without knowing something about the topic of concern beforehand.
I've had several stories assigned to me in the topic of linux installation and configuration e.g. installing a webserver and a mysql database.
Both the webserver and database I've installed several times before and knew exactly which steps I was supposed to do to get it up and running as soon as possible.
Of course this made it really easy to estimate an expected time for these stories, or at least most of them.
One story was about installing Ubuntu.
Even though I have installed Ubuntu several times before I didn't anticipate that an entire installation could be so slow, I had simply missed to take into account that we were running our server on a very old machine.
The time estimate for this story was of by 100\%.

The other stories that didn't go so well was mainly about python programming and working with the coapy server implementation. 
It took generally more time than I expected to get started.
In the future I will increase the expected time before I get going e.g. the time to read about a new software I will work with and the time to try out the current functionality.
Though I will keep my estimate for the actual coding parts.

In the future when it's up to me to do another estimate concerning a topic I've never worked with I hope I have a few colleagues with knowledge in that area to help me out with my first estimate.
As long as I guess an estimate instead of chosing "I don't know" I will improve and eventually I will know my "speed".

\subsubsection{Marcus Rådman}
Under första sprinten var min huvuduppgift att sätta mig in i jcoap, när jag såsmåningom förstå jcoap delvis fösökte jag tillsammans med William att få det att fungera tillsammans med bluetooth.
Att man inte viste hur någonting fungerade gorde det nästan omöjligt att göra en bra tidsestimering. 

När vi såsmåningom skrotade försöket att skicka coap över bluetooth till sprint två var det inga större problem att skicka ett tomt coap paket.
Att kunna skicka paket med innehåll räknade jag skulle ta sex timmar, denna tid inkluderade en bred marginal för oförutsedda problem. Efter nio timmar fungerade det till slut
Med detta i ryggen tog jag itu med att också kunna ta emot medelanden. Detta fungerade nästan "out of the box" tog endast en timme att skriva och testa istället för de beräknade två.
Efter det tog jag itu med en massa småsaket som alldrig blev uppskrivna som en punkt i scrum.

...

Jag tycker att det har varigt lite svårt att arbeta efter SCRUM, 
Det har varit mycket småfix, saker som man måste lära sig hur det fungerar och annat smått och gått som är svåra att skriva in som en punkt i SCRUM.
Det stora problemet har faktiskt varit att jag inte vetat hur android, det har varigt någor man behövt lära sig. 

Om jag fick en chans att göra om projektet från början skulle jag i första a hand insistera på en bättre och nogrannare planering av programstruktuern i android appen.
En annan sak som jag skulle vilja är att vara bättre på att dela upp punkter till mindre punkter.
Att man inte vet hur något fungerar på grund av att man inte arbetat me det tigigare är något som man inte kan göra mycket åt. Väl strukturerat arbete kan däremot begänsa effekten av detta.
\subsubsection{Kristoffer Svensson}
\paragraph{Assignments}
\begin{enumerate}
\item{"Restructure the loading of new services for the python server"}

I had some previous experience with python on top of already been dealing with the actual server implemention we decided to use prior to dealing with this issue so I was pretty certain I would be able to complete it in about 3 hours.
Without any major issues I managed to complete it in roughly 3 hours and 30 minutes. The question at that point was rather if the solution was adequate, which after some consultation it was deemed to be.

\item {"Figure out how service discovery works in CoAP"}

The estimated time for this was set between 8 and 10 hours. The reason being that it was new ground for me but still didn't feel like it was that big of
an issue. It turned out to be a relatively small fix after 6 hours of work. The issue of whether the solution was good enough or not for the end-purposes was brought up again but it was decided that it was good enough for the time being.
With that said, I doubt that the estimated 8-10 hours would have been sufficient if a end-purpose qualifying solution was to be made. 

\item {"Implement EXIficient"}

Originally the time for this assignment was set to 20 hours, when the assignment was still "Implement EXI parser". In that situation we assumed that there was more work to be done making the parser from the ground up.

When EXIficient was discovered we found out that it had a ready-made demo that almost suited our needs. It turned out that the actual time needed to get something that would do the job for us would be around 10 hours modifying that demo.

\item {"Individual documention"}

I assumed about 2-4 hours for the actual writing with an added hour or two for the research needed to know what was to be included in the report. In the end, this all came down to about 4-5 hours which is well within the expected range.

\end{enumerate}

\paragraph{General thoughts concerning time estimation}

I feel as if my personal time estimations, contrary to what I expected, have been quite accurate. I reckon the reason for this being that I had the fortune of having some kind of knowledge concerning the assignments prior to doing them. Without any idea on the scope or size of the assignments I think that the estimations wouldn't have been this accurate.

The estimations themselves have been useful tools for knowing when to ask for help or guidance. When you're approaching a time limit and you know that you're kind of stuck that's a really good indicator that help is needed. I think it's been quite a useful tool, at least for me personally.

\subsubsection{Ludwig Thurfjell}
The project group was divided into three groups, one for C-coding on the Mulle, one for Android phone
and one to set up the server. I was part of the two-man group for the Mulle. We worked with the Scrum model 
for the project, thus there were stories and iterations. 

We had three iterations in total in this project, with sprint planning and what we would like to demo at the 
end of each period. I will show three different stories I've participated in, and so one clearly can see
how the project has matured from day one to the last.

Ideally what I have experianced in the course, is that you need to really break down problems into very small
parts, thus creating many smaller stories which can be significantly easier to estimate the time of. The joy of completing
something is also more rewarding, even though there are small steps to the way of the big picture. It's very demoralizing to 
have a single giant problem to beat.

In the first iteration, we had a story called ''CoAP communication over bluetooth from Mulle to server'', with a huge score
in the estimation part of the sprint planning. Needless to say this story was enormous and involved so much work which at the start
was very hard to imagine. Thus the time estimation of this story was very unstable. Without any knowledge of the CoAP protocol or about the Mulle, this story could take a week or months.

The project changed a bit in term of content and I will jump to describe two stories from the third and last iteration.
One of my stories was ''Run appropriate functions from whatever server calls for''. Since I had read a whole lot about CoAP and 
the LWIP stack, I made a handler quite fast, although it wasn't possible to try it on the Mulle and develop it further since
we couldn't at any stage of the project get simple communication working. Thus the code was scrapped, but the time estimation was somewhat
correct in regards that it would probably have taken lots of time to debug and implement it. 
This story could easily have been broken down to smaller parts as well.

Last story I will describe is ''Internet to Mulle from computer running Bluesoleil''. This one was rated with a low score, which is 
very reasonable. It took less time than expected to complete this story since the installation was straightforward, and the conclusion
that it doesn't work at all was quickly made.

\subsection{Reflection about Scrum usage during this project}
A lot has been learnt during this project and to keep it short the following lessons learnt and improvements that can be made are a chosen few.

At the end of all projects when the deadline is closing in the pace of the development often increases.
A downfall of this is that when the speed increases the quality of the code often decreases.
Scrum is a solution to this problem with the main goal of keeping a steady development pace and always keeping the quality of the code as good as possible above some minimum criteria.
With too long sprints, projects will still end up with a big deadline, this is what happened in this project.
In total the project lasted about 17 weeks including the winter holidays.
Instead of three sprints where each lasted about five weeks a project split into smaller sprint would have been better.
If time travel was possible this project would have had two smaller sprints in the beginning each would have lasted for two weeks.
The first sprint with goal of configuring all required software for everyone such as setting up git, the different IDEs, gcc and other required software/tools.
The second sprint with the goal of understanding what was available at the time and get all basic functionality working.
If the project had found any big bottlenecks at the end of the second sprint that would be the point in time to halt the project and really rethink what to focus on.
The remaining weeks should have been divided into three evenly sized sprints.

Another big improvement could have been made in the relation between Scrum team and product owner.
From the beginning it was unclear who was the product owner out of three supervisors/clients, this should have been defined to one single person as early as possible before continuing with any other work.
The absence of the product owner before and during our sprint planning meetings resulted in a lot of confusion when it was up to the team to decide an estimate for each sprint story.
The lesson learnt from this is that without a product owner the scrum team will fumble in blindness forever or as Kniberg writes \cite[p.~25]{kniberg07} {\em "...each story contains three variables that are highly dependent on each other"}.
There is no way the team can choose a good estimate when there is no product owner that has already chosen an importance and scope for each sprint story to start with and is available to change that during the course of a sprint planning meeting.
This also lead to the project growing in scope without any clear bounds.

The last and perhaps the most important improvement that will be mentioned is the importance of having a team member taking the role as Scrum master.
Without the Scrum master during the first sprint everyone was on their own.
For the second and third sprint the Scrum master role was appointed to one team member.
This improved the communication within the Scrum team a lot and also made it possible for individual team members to have someone to go to in case of general questions and/or other problems.

%
% Andra kommentarer som kan vara värda att ta med om Scrum och hur det 
% har använts i detta projekt om det får plats. Texterna är redan lite för långa.
%
%TODO: Tydligare sprint mål för varje sprint från kravställare. En mening som beskriver syftet med respektive sprint.
%TODO: Tydligare demo av varje punkt
%TODO: Vi skulle ha satt ner foten på hur många delar vi skulle jobba med samtidigt. Startat med Mulle och Server enbart...sedan android i mån av tid för att få mer fokus i projektet, vi var trots allt bara sju personer.
%TODO: Ingen "experthjälp" med android tillgänglig. I detta läge gör det stor skillnad om man är två, tre eller fyra. Varje programmerare ökar hastigheten med mer än bara sin tid hen kan lägga till eftersom felsökning oftast går betydligt snabbare för fler personer än enbart två. Många fel beror också på utvecklingsmiljön, om man är fler med samma miljlö så är det större chans att någon har stött på problemet tidigare och har en lösning.
%TODO: Reflektion över flytten av individuella team members server till mulle och android. Väldigt "dyrt" i uppstartskostnad.
